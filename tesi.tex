\documentclass[a4paper,11pt,twoside]{book}
\usepackage[DIV=14,BCOR=2mm,headinclude=true,footinclude=false]{typearea}
\usepackage[activate={true,nocompatibility},final,tracking=true,kerning=true,spacing=true,factor=1100,stretch=10,shrink=10]{microtype}
\usepackage[english]{babel}
\usepackage{enumerate}
\usepackage{longtable}
\usepackage{makecell}
\usepackage{comment}
\usepackage{tikz-feynman}
\usepackage{float}
\usepackage{graphicx}
\usepackage[format=plain,labelfont=it, textfont=it]{caption}
\usepackage{subcaption}
\usepackage{booktabs}
\usepackage{amsmath}
\usepackage{mathrsfs}

\counterwithin{figure}{chapter} 
\counterwithin{equation}{section}





%\usepackage[backend=bibtex,
%style=numeric-comp, sorting=none, block=ragged, giveninits=true, ]{biblatex}

%\addbibresource{bibliography/bib/mybiblio_2.bib}

\usepackage{hyperref}
\usepackage{cleveref}
\hypersetup{
colorlinks=true,
linkcolor=blue,
filecolor=magenta,
citecolor=red,
urlcolor=blue,
pdftitle={nico thesis},
pdfpagemode=FullScreen,
}



\setlength{\oddsidemargin}{8mm}   
\setlength{\textwidth}{145mm}     
\linespread{1.12}                


\renewcommand\labelitemi{\tiny$\bullet$}
\newcommand{\isotope}[2]{\textsuperscript{#1}#2}




\begin{document}
\pagenumbering{gobble}



\begin{titlepage}
	
\begin{figure}[H]
	\centering
	\includegraphics[width=360pt]{logo_2.jpg}
	\vspace{0.8 cm}
\end{figure}

	
	
\begin{center}{
Facoltà di Scienze e Tecnologie\\
Laurea Triennale in Fisica
}
\end{center}
	
\begin{center}
\vspace{1 cm}
{
\textbf{Near-Infrared Extinction}}
\end{center}
\par
\vspace{2 cm}
	
\begin{flushleft}
Relatore:\\ \textbf{Marco Lombardi}\\
\end{flushleft}
	
\vspace{1 cm}
\begin{flushright}
Tesi di Laurea Triennale di:\\ \textbf{Edoardo Tedesco} \\Matricola: \textbf{933636}
\end{flushright}
	
\vfill
\begin{center}
{\large Anno Accademico 2022/2023}
\end{center}
\end{titlepage}



\cleardoublepage\thispagestyle{empty} 
\vspace*{1cm} 
\textit{
\flushright
alla mia famiglia
\vfill
}



\chapter*{\centering Abstract}
\begin{center}


The study of the top quark is essential for the ATLAS (\textbf{A} \textbf{T}oroidal \textbf{L}HC \textbf{A}pparatu\textbf{S})  experiment, because it could help shedding a light over the electroweak symmetry breaking mechanism. \\
The top quark was discovered at Tevatron in 1995. \\
It is the heaviest particle of the Standard Model (SM) with a coupling to the Higgs boson close to one.
The overwhelming amount of data collected by the ATLAS Experiment at the LHC at CERN during Run-II has allowed the recording and measuring of rare processes, foreseen by the SM, that have never been observed before.
Recently, the ATLAS Experiment has registered a sporadic process of primary importance, which involves the top quark: the production of a single top quark in association with a Z boson (tZq event).
The cross section of this process, predicted by the SM to be 102 fb, has been measured with an uncertainty of 15\%.
Since then the aim has been to find new ways to measure the process, decreasing the uncertainty.
The purpose of this thesis is to investigate Machine Learning techniques which would improve the discrimination of tZq events from background sources in proton-proton collisions at $\sqrt{s}$ = 13 TeV in the ATLAS experiment. \\
In the first and second chapter I will provide an overview on the Standard Model and the top quark physics.
In the third chapter I will give a brief description of LHC and the ATLAS experiment.
In the fourth chapter I will give a short introduction on Machine Learning.
In the fifth and last chapter I will explain the work I have done in the past few months.

\end{center}


\newpage\null\thispagestyle{empty}\newpage
\tableofcontents
\newpage\null\thispagestyle{empty}\newpage

\pagenumbering{arabic}
\mainmatter

%primo capitolo
\chapter{The Standard Model}



%secondo capitolo
\chapter{Top quark}



%terzo capitolo
\chapter{The Large Hadron Collider and the ATLAS Experiment}



%quarto capitolo
\chapter{Machine Learning}


%quinto capitolo
\chapter{Separation of tZq from background with DNNs} 



\backmatter
\addcontentsline{toc}{chapter}{Bibliography}
%\printbibliography
\end{document}

